\documentclass[aspectratio=169]{beamer}
\usetheme{Madrid}
\usecolortheme{default}

\usepackage[utf8]{inputenc}
\usepackage[vietnamese]{babel}
\usepackage{graphicx}
\usepackage{hyperref}
\usepackage{listings}
\usepackage{xcolor}

% Code styling
\lstset{
    basicstyle=\ttfamily\small,
    breaklines=true,
    frame=single,
    backgroundcolor=\color{gray!10}
}

\title{eShelf - Enterprise eBook Platform}
\subtitle{Đồ án môn học IE104 - DevOps \& MLOps}
\author{Nhóm eShelf}
\institute{Đại học Công nghệ Thông tin (UIT)}
\date{\today}

\begin{document}

% Title slide
\frame{\titlepage}

% Table of contents
\begin{frame}
\frametitle{Mục lục}
\tableofcontents
\end{frame}

% Section 1: Giới thiệu
\section{Giới thiệu}

\begin{frame}
\frametitle{Tổng quan dự án}
\begin{itemize}
    \item \textbf{eShelf} - Nền tảng đọc sách điện tử enterprise-grade
    \item Kiến trúc \textbf{Microservices} với 4 services chính
    \item Áp dụng đầy đủ quy trình \textbf{DevOps} và \textbf{MLOps}
    \item Tech stack hiện đại: React, Node.js, FastAPI, PostgreSQL, Kubernetes
\end{itemize}
\end{frame}

\begin{frame}
\frametitle{Tính năng chính}
\begin{columns}
\column{0.5\textwidth}
\textbf{Người dùng:}
\begin{itemize}
    \item Đọc sách PDF trực tuyến
    \item Tìm kiếm và lọc sách
    \item Quản lý bộ sưu tập
    \item Gợi ý sách AI-powered
    \item Theo dõi tiến độ đọc
\end{itemize}

\column{0.5\textwidth}
\textbf{Admin:}
\begin{itemize}
    \item Dashboard thống kê
    \item Quản lý sách (CRUD)
    \item Quản lý người dùng
    \item Quản lý thể loại
\end{itemize}
\end{columns}
\end{frame}

% Section 2: Kiến trúc
\section{Kiến trúc hệ thống}

\begin{frame}
\frametitle{Microservices Architecture}
\begin{center}
\includegraphics[width=0.9\textwidth]{architecture-diagram.png}
\end{center}
\begin{itemize}
    \item \textbf{API Gateway} (Port 3000): Routing, Rate Limiting
    \item \textbf{4 Services}: Auth, Book, User, ML
    \item \textbf{Database}: PostgreSQL + Redis
\end{itemize}
\end{frame}

\begin{frame}
\frametitle{Services Overview}
\begin{table}[h]
\centering
\small
\begin{tabular}{|l|l|l|}
\hline
\textbf{Service} & \textbf{Port} & \textbf{Technology} \\
\hline
Frontend & 5173 & React + Vite \\
API Gateway & 3000 & Express.js \\
Auth Service & 3001 & Express.js \\
Book Service & 3002 & Express.js \\
User Service & 3003 & Express.js \\
ML Service & 8000 & FastAPI (Python) \\
PostgreSQL & 5432 & PostgreSQL 16 \\
Redis & 6379 & Redis 7 \\
\hline
\end{tabular}
\end{table}
\end{frame}

% Section 3: Tech Stack
\section{Tech Stack}

\begin{frame}
\frametitle{Frontend \& Backend}
\textbf{Frontend:}
\begin{itemize}
    \item React 18 + Vite
    \item TailwindCSS
    \item React Router, Recharts
\end{itemize}

\vspace{0.5cm}
\textbf{Backend:}
\begin{itemize}
    \item Node.js 20 + Express.js
    \item Prisma ORM
    \item JWT Authentication
\end{itemize}
\end{frame}

\begin{frame}
\frametitle{ML/AI \& DevOps}
\textbf{ML/AI:}
\begin{itemize}
    \item FastAPI (Python)
    \item scikit-learn, numpy, pandas
    \item Collaborative Filtering, Content-based Filtering
\end{itemize}

\vspace{0.5cm}
\textbf{DevOps:}
\begin{itemize}
    \item Docker, Docker Compose
    \item Kubernetes (EKS/K3s)
    \item Terraform, CloudFormation
    \item GitHub Actions, Jenkins
    \item ArgoCD (GitOps)
\end{itemize}
\end{frame}

% Section 4: CI/CD Pipeline
\section{CI/CD Pipeline}

\begin{frame}
\frametitle{CI/CD Flow}
\begin{enumerate}
    \item \textbf{Source} → Pull Request
    \item \textbf{CI (PR checks)}: lint → unit test → typecheck → static analysis
    \item \textbf{Image Build \& Scan}: multi-stage Docker build → Trivy scan → push to registry
    \item \textbf{Infrastructure}: terraform plan/apply (staging)
    \item \textbf{Deploy Staging}: deploy to K8s → integration/e2e tests
    \item \textbf{Promote to Prod}: manual approval → blue/green deploy → smoke tests
    \item \textbf{GitOps}: ArgoCD sync from Git repo
    \item \textbf{Rollback}: automatic on failing healthchecks
\end{enumerate}
\end{frame}

\begin{frame}
\frametitle{Smart Build}
\textbf{Vấn đề:}
\begin{itemize}
    \item Trong Microservices, sửa 1 service không nên build lại toàn bộ
\end{itemize}

\vspace{0.3cm}
\textbf{Giải pháp:}
\begin{itemize}
    \item Path-filter trong CI/CD pipeline
    \item Chỉ build service có thay đổi
    \item Tối ưu thời gian build và tài nguyên
\end{itemize}

\vspace{0.3cm}
\textbf{Ví dụ:}
\begin{lstlisting}[language=bash]
if [ -d "backend/services/auth-service" ]; then
    build auth-service
fi
\end{lstlisting}
\end{frame}

\begin{frame}
\frametitle{GitOps \& Image Tagging}
\textbf{Quy trình:}
\begin{enumerate}
    \item GitHub Actions build Docker Image
    \item Push image lên Harbor/ECR với tag mới
    \item Dùng \texttt{yq} hoặc \texttt{kustomize} update image tag trong Git repo
    \item ArgoCD phát hiện thay đổi và sync về cluster
    \item Tự động deploy version mới
\end{enumerate}

\vspace{0.3cm}
\textbf{Lợi ích:}
\begin{itemize}
    \item Không cần \texttt{kubectl apply} thủ công
    \item Tất cả thay đổi được track trong Git
    \item Dễ rollback và audit
\end{itemize}
\end{frame}

% Section 5: Infrastructure
\section{Infrastructure}

\begin{frame}
\frametitle{Infrastructure as Code}
\textbf{Terraform:}
\begin{itemize}
    \item VPC với Public/Private Subnets
    \item NAT Gateway cho Private Subnets
    \item EC2 instances (3 nodes: 1 Master, 2 Workers)
    \item Security Groups
    \item EKS hoặc K3s cluster
\end{itemize}

\vspace{0.3cm}
\textbf{Ansible:}
\begin{itemize}
    \item Configuration Management
    \item Deploy K3s lên 3 môi trường (Dev, Staging, Prod)
    \item Setup Harbor, ArgoCD, monitoring tools
\end{itemize}
\end{frame}

\begin{frame}
\frametitle{Monitoring \& Observability}
\textbf{Stack:}
\begin{itemize}
    \item \textbf{Prometheus}: Metrics collection
    \item \textbf{Grafana}: Visualization dashboards
    \item \textbf{Loki}: Log aggregation
    \item \textbf{Alertmanager}: Alerting
\end{itemize}

\vspace{0.3cm}
\textbf{Metrics theo dõi:}
\begin{itemize}
    \item Service health và uptime
    \item Request rate và latency
    \item Error rates
    \item Resource usage (CPU, Memory)
    \item ML model performance
\end{itemize}
\end{frame}

% Section 6: MLOps
\section{MLOps}

\begin{frame}
\frametitle{ML Service}
\textbf{Tính năng:}
\begin{itemize}
    \item \textbf{Recommendations}: Gợi ý sách dựa trên Collaborative Filtering
    \item \textbf{Similar Books}: Tìm sách tương tự dựa trên nội dung
    \item \textbf{Reading Time Estimation}: Ước tính thời gian đọc
\end{itemize}

\vspace{0.3cm}
\textbf{MLOps Pipeline (Planned):}
\begin{enumerate}
    \item Model training CI
    \item Model registry (MLflow)
    \item Model packaging
    \item Canary deploy
    \item Monitoring metrics \& data drift
\end{enumerate}
\end{frame}

% Section 7: Security
\section{Security}

\begin{frame}
\frametitle{Security Measures}
\textbf{Application:}
\begin{itemize}
    \item JWT authentication với refresh tokens
    \item Password hashing (bcrypt, 12 rounds)
    \item Rate limiting (100 requests/15min)
    \item CORS configuration
    \item Security headers (Helmet)
    \item Input validation
\end{itemize}

\vspace{0.3cm}
\textbf{Infrastructure:}
\begin{itemize}
    \item Security Groups (chỉ cho phép IP cụ thể)
    \item Private Subnets với NAT Gateway
    \item Container scanning (Trivy)
    \item IaC scanning (Checkov)
    \item Code quality (SonarQube)
\end{itemize}
\end{frame}

% Section 8: Demo
\section{Demo}

\begin{frame}
\frametitle{Demo Features}
\begin{enumerate}
    \item \textbf{User Flow}: Đăng ký → Đăng nhập → Tìm sách → Đọc sách
    \item \textbf{Admin Flow}: Dashboard → Quản lý sách → Quản lý users
    \item \textbf{ML Features}: Recommendations → Similar books
    \item \textbf{CI/CD}: Show pipeline logs → Smart build demo
    \item \textbf{GitOps}: Show ArgoCD sync
    \item \textbf{Monitoring}: Show Grafana dashboards
\end{enumerate}
\end{frame}

% Section 9: Kết quả
\section{Kết quả \& Đánh giá}

\begin{frame}
\frametitle{Đã hoàn thành}
\textbf{Application:}
\begin{itemize}
    \item ✅ 4 Microservices hoạt động ổn định
    \item ✅ Frontend đầy đủ tính năng
    \item ✅ ML Service với 3 APIs
    \item ✅ Admin panel hoàn chỉnh
\end{itemize}

\vspace{0.3cm}
\textbf{DevOps:}
\begin{itemize}
    \item ✅ Docker Compose setup
    \item ✅ CI/CD pipeline cơ bản
    \item ✅ Terraform infrastructure
    \item ✅ Kubernetes manifests
\end{itemize}
\end{frame}

\begin{frame}
\frametitle{Đang phát triển}
\textbf{CI/CD:}
\begin{itemize}
    \item 🔄 Smart Build với path-filter
    \item 🔄 ArgoCD Image Updater
    \item 🔄 Blue/Green deployment
    \item 🔄 Automated rollback
\end{itemize}

\vspace{0.3cm}
\textbf{Infrastructure:}
\begin{itemize}
    \item 🔄 AWS deployment (EKS/K3s)
    \item 🔄 Harbor/Artifactory setup
    \item 🔄 Complete monitoring stack
    \item 🔄 Ansible scripts
\end{itemize}
\end{frame}

% Section 10: Hướng phát triển
\section{Hướng phát triển}

\begin{frame}
\frametitle{Hướng phát triển tương lai}
\textbf{Features:}
\begin{itemize}
    \item Reviews \& Ratings system
    \item Notifications system
    \item Advanced search filters
    \item Social features (sharing, comments)
\end{itemize}

\vspace{0.3cm}
\textbf{Technical:}
\begin{itemize}
    \item Redis caching implementation
    \item E2E testing với Playwright
    \item Load testing
    \item MLflow integration
    \item Model versioning và monitoring
\end{itemize}
\end{frame}

\begin{frame}
\frametitle{Rollback Strategy}
\textbf{Khi deploy lỗi:}
\begin{enumerate}
    \item Healthcheck fails → tự động rollback
    \item ArgoCD phát hiện và revert về version cũ
    \item Hoặc manual rollback qua ArgoCD UI
    \item Audit logs được lưu lại
\end{enumerate}

\vspace{0.3cm}
\textbf{Blue/Green Deployment:}
\begin{itemize}
    \item Deploy version mới (Green) song song với version cũ (Blue)
    \item Test Green environment
    \item Switch traffic sang Green
    \item Nếu lỗi → switch lại Blue ngay lập tức
\end{itemize}
\end{frame}

% Section 11: Kết luận
\section{Kết luận}

\begin{frame}
\frametitle{Kết luận}
\begin{itemize}
    \item Dự án đã xây dựng thành công hệ thống \textbf{enterprise-grade} với kiến trúc microservices
    \item Áp dụng đầy đủ quy trình \textbf{DevOps} và \textbf{MLOps}
    \item Tuân thủ best practices về security, monitoring, và CI/CD
    \item Sẵn sàng scale và deploy lên production
\end{itemize}

\vspace{0.5cm}
\textbf{Điểm mạnh:}
\begin{itemize}
    \item Kiến trúc rõ ràng, dễ maintain
    \item CI/CD pipeline tự động hóa
    \item Security và monitoring đầy đủ
    \item Code quality tốt
\end{itemize}
\end{frame}

\begin{frame}
\frametitle{Cảm ơn}
\begin{center}
\Large
\textbf{Cảm ơn thầy/cô và các bạn đã lắng nghe!}

\vspace{1cm}
\textbf{Q \& A}
\end{center}
\end{frame}

\end{document}

